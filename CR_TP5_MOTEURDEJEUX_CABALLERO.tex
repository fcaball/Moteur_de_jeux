

\documentclass{article}
\usepackage[utf8]{inputenc}
\usepackage[utf8]{inputenc}
\usepackage[T1]{fontenc}
\usepackage[english]{babel}
\usepackage{fullpage}
\usepackage{color}
\usepackage[table]{xcolor}
\usepackage{listings}
 
\definecolor{darkWhite}{rgb}{0.94,0.94,0.94}
 
\lstset{
  aboveskip=3mm,
  belowskip=-2mm,
  backgroundcolor=\color{darkWhite},
  basicstyle=\footnotesize,
  breakatwhitespace=false,
  breaklines=true,
  captionpos=b,
  commentstyle=\color{red},
  deletekeywords={...}, 
  escapeinside={\%*}{*)},
  extendedchars=true,
  framexleftmargin=16pt,
  framextopmargin=3pt,
  framexbottommargin=6pt,
  frame=tb,
  keepspaces=true,
  keywordstyle=\color{blue},
  language=C,
  literate=
  {²}{{\textsuperscript{2}}}1
  {⁴}{{\textsuperscript{4}}}1
  {⁶}{{\textsuperscript{6}}}1
  {⁸}{{\textsuperscript{8}}}1
  {€}{{\euro{}}}1
  {é}{{\'e}}1
  {è}{{\`{e}}}1
  {ê}{{\^{e}}}1
  {ë}{{\¨{e}}}1
  {É}{{\'{E}}}1
  {Ê}{{\^{E}}}1
  {û}{{\^{u}}}1
  {ù}{{\`{u}}}1
  {â}{{\^{a}}}1
  {à}{{\`{a}}}1
  {á}{{\'{a}}}1
  {ã}{{\~{a}}}1
  {Á}{{\'{A}}}1
  {Â}{{\^{A}}}1
  {Ã}{{\~{A}}}1
  {ç}{{\c{c}}}1
  {Ç}{{\c{C}}}1
  {õ}{{\~{o}}}1
  {ó}{{\'{o}}}1
  {ô}{{\^{o}}}1
  {Õ}{{\~{O}}}1
  {Ó}{{\'{O}}}1
  {Ô}{{\^{O}}}1
  {î}{{\^{i}}}1
  {Î}{{\^{I}}}1
  {í}{{\'{i}}}1
  {Í}{{\~{Í}}}1,
  morekeywords={*,...},
  numbers=left,
  numbersep=10pt,
  numberstyle=\tiny\color{black},
  rulecolor=\color{black},
  showspaces=false,
  showstringspaces=false,
  showtabs=false,
  stepnumber=1,
  stringstyle=\color{gray},
  tabsize=4,
  title=\lstname,
}
\usepackage{graphicx}
\title{HAI819I – Moteurs de jeux
}
\author{Fabien Caballero}

\begin{document}

\maketitle
    \tableofcontents

\newpage
\section{Mouvement progressif}

Pour la première question, j'ai pour chaque frame translaté mon objet d'un vecteur vitesse déclaré plus haut multiplié par le temps écoulé depuis la dernière frame.
Ainsi on obtient un mouvement progressif dépendant du temps.

\section{Ajout du poids}

Afin d'ajouter le poids on calcule la force de gravité, pour cela on définit une masse à notre objet, que l'on multiplie par l'accéleration de gravité.
Cette force est représentée par un vecteur, dans notre cas on applique le résultat -(m*g) en y pour simuler la pesanteur, en x et en z étant donné qu'aucune autre force n'est appliquée on va avoir 0.

Ensuite on ajoute à la vitesse le vecteur accéleration, qui est la force calculée précédemment divisée par la masse, multiplié par deltaTime (le temps entre 2 frames) ainsi notre objet chute progressivement.

\section{Arrêter l'objet au contact du plan}
Pour cela, on définit un centre ou la boite englobante de notre objet puis on compare les y et lorsque le y de notre objet atteint le y de notre plan on met notre vitesse à 0, ainsi l'objet s'arrête.

\section{Rebond}

Pour la suite nous souhaitons ajouté un rebond, pour cela à la place de mettre la vitesse à 0 comme précédemment on fait un rebond à la place.

Pour cela on multiplie la vitesse par -1 afin que le vecteur pointe dans l'autre sens puis on remultiplie le x du vecteur vitesse par -1 afin que l'objet avance et ne recule pas.

\section{Bonus: Il n'y a plus de sol !}
J'ai par la suite ajoutée une fonctionnalité qui fait tombé notre objet s'il n'est plus sur le plan.
Il suffit de faire le même principe que ce que l'on a fait pour savoir si l'objet touche le plan, mais cette fois-ci selon l'axe x, afin de savoir si l'objet a dépassé le plan (ceci est facilement adaptable pour toutes les directions du plan).
Si l'on a dépassé le plan on ne fait pas les même actions que précédemment mais on déplace juste notre objet vers le bas, en utilisant la force de gravité.
Ainsi lorsque l'objet dépasse le plan il va simplement être affecté par la gravité et tomber vers le bas.



\end{document}